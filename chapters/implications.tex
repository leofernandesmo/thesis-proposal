\chapter{Implications for Practice}
\label{chp:implications}

The high cost of mutation testing is well known. 
Our rules can contribute to the development of better tools and to decrease costs in mutation testing analysis. 
In the end, users can benefit from mutation tools that generate less useless mutants.

In addition, as programming languages evolve, new constructs are added, e.g., Java lambda. 
In this way, developers of mutation testing tools tend (i) to create new mutation operators to cover these new language constructs and (ii) to evolve the existing operators. 
Therefore, they could use our strategy to derive and implement rules to deal with both situations. 
Thus, before releasing the tool, developers can improve confidence that the new version can avoid useless mutants when considering the new operators and the evolved ones. 
However, differently from our evaluation, developers would focus on specific operators (not all available in the tool), reducing costs on carrying out our strategy.

Besides new constructs, the mutation system developer can use the strategy to check if the transformation for a specific mutation operator is working as expected with a wide range of programs. 
During the experiment, our strategy found transformation bugs in \mujava{} and \pit{} that were reported to the respective developers.

%\todo{Leo, precisamos pensar e adicionar mais coisas aqui!}

