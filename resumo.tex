Teste de Mutação é uma técnica de transformação de programa que injeta falhas artificiais para checar se a suíte de testes existente consegue detectá-la. Contudo, o custo de usar teste de mutação é muito alto, o que dificulta seu uso pela indústria. Mutantes inúteis (equivalentes e duplicados) contribuem para aumentar os custos. Um mutante equivalente é um mutante que possui o mesmo comportamento do programa original, o que o torna inútil. Uma descoberta mais recente são os mutantes que possuem o mesmo comportamento que outros mutantes, chamados de duplicados. Desta forma, um deles é inútil. Pesquisas anteriores focaram principalmente em detectar mutantes inúteis apenas após eles serem gerados e compilados. Neste artigo, nós introduzimos uma estratégia para ajudar desenvolvedores a derivar regras (RULES) para evitar a geração destes mutantes inútei. Para usar nossa estratégia, fornecemos um conjunto de programas. Para cada programa, nós passamos uma suíte de testes e um conjunto de mutantes. O resultado da nossa estratégia é um conjunto de mutantes inúteis. Após manualmente confirmar que os mutantes classificados pela nossa estratégia eram realmente inúteis, nós derivamos regras que evitam sua geração e com isso reduz o custo. Em particular, até onde sabemos, nós contibuímos com \NumberOfNewHeuristics novas regras que podem evitar mutantes inúteis antes deles serem gerados. Nós, então, implementamos um subconjunto destas regras na ferramenta de teste de mutação \mujava{}. Visto que nossas regras foram derivadas programas pequenos e artificiais, nós executamos esta nova versão do \mujava{}, com as novas regras, e executamos em projetos reais. Nossas regras reduziram o número de mutantes em aproximadamente 13\% em média. Nossos resultados são promissores porque (i) nós \textit{evitamos} a geração de mutantes inúteis; (ii) nossa estratégia pode ajudar a identificar mais regras no caso de nós passarmos programas Java mais complexos; e (iii) nossa versão do \mujava{} possui apenas um subconjunto da regras que derivamos.

\begin{keywords}
	Engenharia de Software, Teste de Software, Teste de Mutação, Mutantes Inúteis
\end{keywords}